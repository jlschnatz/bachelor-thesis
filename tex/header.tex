\usepackage{unicode-math}

\usepackage[english]{babel} 
\raggedbottom               % do not spread content to bottom of page
\linespread{1.5}            % global line spread
\usepackage{setspace}       % to modify line spread within the document
\usepackage[nottoc]{tocbibind} % puts lof & lot, but not toc itself in toc
\usepackage[section]{placeins} % keep floats in respective section
\usepackage{indentfirst}    % indent first paragraph also
\usepackage{sectsty}        % header styling
\sectionfont{\fontsize{12}{12}\bfseries\centering} % section header
\subsectionfont{\fontsize{12}{12}\bfseries}        % subsection header 
\subsubsectionfont{\fontsize{12}{12}\bfseries\itshape}  % subsubsection 
\usepackage{caption}                                    % caption styling
\DeclareCaptionLabelSeparator*{spaced}{\\[1ex]}
\captionsetup[table]{labelfont=bf, textfont=it, format=plain, justification=justified, singlelinecheck=false, labelsep=spaced, skip=12pt}
\captionsetup[figure]{labelfont=bf, textfont=it, labelsep=spaced, justification=justified, singlelinecheck=false, labelsep=spaced, skip=12pt}
\usepackage{fancyhdr}                 % fancy headers and footers
\setlength{\headheight}{12pt}
\pagestyle{fancy}
\fancyhf{}
\fancyhead[LE,RO]{\thepage}
\fancyhead[RE,LO]{}

\usepackage{tabularx} % automatic line break in a table

\addtokomafont{disposition}{\rmfamily} % use roman font for section headers

\usepackage{float} % for figures 
\floatplacement{figure}{H} % force figures to be placed where they are in the code
\usepackage[fontsize=11.5pt]{fontsize} % change font size
\usepackage{caption} % caption styling
\captionsetup{format = plain, labelfont=bf, labelsep = newline} % newline for caption after bold title

\setlength{\textfloatsep}{10pt plus 3pt minus 5pt}
\usepackage{ragged2e}
\usepackage{parskip}
\parindent=0.5in
\setlength{\headsep}{10pt}
\setlength{\parskip}{0pt}

\usepackage{units} % for nicefrac

% remove dots from toc
\usepackage[titles]{tocloft}
\renewcommand{\cftdot}{}

% nsum definition
\usepackage{calc}
\newlength{\depthofsumsign}
\setlength{\depthofsumsign}{\depthof{$\sum$}}
\newlength{\totalheightofsumsign}
\newlength{\heightanddepthofargument}
\newcommand{\nsum}[1][1.6]{
  \mathop{
    \raisebox
    {-#1\depthofsumsign+1\depthofsumsign}
    {\scalebox
      {#1}
      {$\displaystyle\sum$}
    }
  }
}

% shortcuts for publication bias parameter
\newcommand{\pbs}{\omega_{\text{PBS}}}
\newcommand{\epbs}{\widehat{\omega}_{\text{PBS}}}

\usepackage{tabu} % tabu env for tables
\usepackage{multirow} % valign of table cells
\usepackage{longtable} % longtable
\usepackage{threeparttablex} % threeparttable
%\usepackage{amsmath} % math font

% shortcut for hypothesis naming [number hypothesis, type hypothesis (null: 0, alternative: 1)]
\newcommand{\hypothesis}[2]{\mathcal{H}^{\text{(#1)}}_{\text{#2}}}

% use biblatex for references with apa-7 format
%\usepackage[style=apa, backend=biber]{biblatex}
%\addbibresource{../bibliography/tidy_references.bib} % add references to header
%\DeclareFieldFormat{titlecase}{#1}
%\AtEveryBibitem{\clearfield{note}\clearfield{urlyear}\clearfield{urlmonth}\clearfield{urlday}} % exclude notes and urldate

%\usepackage[backend=biber,natbib=true,style=apa]{biblatex} 

\usepackage{appendix}

\usepackage{hyperref}

\usepackage{amsmath}        % more math mode options
%\usepackage[bb=ams]{mathalpha}
\usepackage{bm}             % bold font in math mode
